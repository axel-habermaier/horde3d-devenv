\section{Konzeptmodell}

Abbildung~\ref{fig:domainModel} zeigt das Konzeptmodell des \DevEnvs. Die meisten Konzepte, ihre Abh�ngigkeiten und Spezialisierungshierarchien wurden von der Struktur und vom Aufbau von \Horde\ vorgegeben. Das Modell musste nur um die Konzepte \texttt{Horde3DApplication}, \texttt{FunctionCall} und \texttt{EditableResource} erweitert werden. Ersteres repr�sentiert die Anwendung, die vom Tool gestartet wird. \texttt{FunctionCall} wurden gegen�ber den System-Anforderungen noch um die Attribute \texttt{ReturnValue} und \texttt{Parameters} erweitert. In diesen beiden Attributen werden die Werte, die an die Funktion �bergeben beziehungsweise von ihr zur�ckgegeben werden, aufgezeichnet. Diese Daten wurden erst in einer sp�teren Iteration ins Konzeptmodell aufgenommen. Damit wird eine der in Kapitel~\ref{Ausblick} beschriebenen Erweiterungen bereits vorbereitet.

Im Konzeptmodell wurden alle Typen von \emph{Scene Nodes} als Spezialisierungen des abstrakten Konzepts \texttt{SceneNode} dargestellt. Bei den Ressourcen wurde zus�tzlich noch das Konzept der editierbaren Ressource hinzugef�gt, da das \DevEnv\ nur gewisse Ressourcentypen -- Materials, Pipelines, Shaders, Particle Effects und Code Ressourcen -- ver�ndern kann. Alle editierbaren Ressourcen sind durch das abstrakte Konzept \texttt{EditableResource} generalisiert, w�hrend alle anderen Ressourcen und \texttt{EditableResource} selbst Spezialisierungen des allgemeineren abstrakten Konzepts \texttt{Resource} sind.

Es wurden au�erdem die Beziehungen der einzelnen Ressourcen untereinander untersucht. Die Pipeline Konfiguration wurde auf einzelne Konzepte aufgeteilt, um m�glichst flexibel auf �nderungen und Erweiterungen der Pipeline in neuen \Horde-Versionen reagieren zu k�nnen. Au�erdem wurden die Assoziationen zwischen den \emph{Scene Nodes} und ihren ben�tigten Ressourcen eingezeichnet. Damit wird es sp�ter m�glich sein, f�r einen \emph{Scene Node} die verwendeten Ressourcen zu betrachten beziehungsweise f�r eine Ressource herauszufinden, von welchen Teilen des Szenengraphs sie verwendet wird. Diese detaillierte Analyse der Assoziationen zwischen den Konzepten war zwar nicht aufgrund der \emph{Use Cases} erforderlich, verbesserte aber das Verst�ndnis des Problembereichs und wird f�r einige der vorgeschlagenen Erweiterungen in Kapitel~\ref{Ausblick} ben�tigt.

Die \Horde-Konzepte, ihre Attribute und ihre Abh�ngigkeiten untereinander wurden aus der API-Beschreibung der \Horde-Dokumentation \cite["`Engine API Reference"']{h3dmanual} ermittelt. Da w�hrend der Entwicklung der Anwendung eine neue \Horde-Version erschien, mussten in einer sp�teren Iteration einige wenige Details der Konzepte ver�ndert werden. Hierbei machte sich allerdings die starke Modularisierung des Konzeptmodells positiv bemerkbar.