\section{Konklusion}
In der Design-Phase wurden viele generische Konzepte verwendet, um flexibel auf m�gliche �nderungen von \Horde\ oder eventuell auftauchende Probleme in der Implementierungs-Phase reagieren zu k�nnen. Aufgrund der detaillierten Analyse der Anforderungen und dem Aufbau von \Horde\ konnte ein Systemdesign erstellt werden, das bei der Implementierung nur wenige Probleme verursachte. Die notwendigen �nderungen am Design beschr�nkten sich jedoch immer auf einzelne Klassen; die grundlegende Architektur blieb bestehen. Eine gr��ere �nderung ergab sich lediglich durch die Einf�hrung der Client-Server-\emph{Callbacks}. 

Im Rahmen dieser Bachelorarbeit wurde aber nicht das komplette Design implementiert. So gibt es gerade bei den \Horde-Klassen einige Attribute und Assoziationen, die zwar im Designmodell vorhanden sind, f�r die Umsetzung der Systemanforderungen aber nicht erforderlich waren. Sie wurden dennoch in das Konzept- und Designmodell aufgenommen, um die Modelle zu vervollst�ndigen. Im Code k�nnen diese bei Bedarf einfach hinzugef�gt werden.

Bei der Entwicklung des \DevEnvs\ wurde auch deutlich, dass der verwendete DLL-\emph{Replacement}-Mechanismus sowie die Einf�hrung der \texttt{Horde3DCall}-Klasse richtige Entscheidungen waren. Insbesondere die beiden Klassen \texttt{Horde3DMessagesHandler} und \texttt{Horde3DStateWatcher} zeigten, warum das Ausl�sen von generischen als auch spezifischen Ereignissen mit den genauen Aufrufsparametern und R�ckgabewerten nach einem \Horde-Funktionsaufruf sinnvoll ist. Diese Informationen sind f�r unterschiedlichste Aktionen n�tzlich; so wurde beim Entwurf dieses Verfahrens nicht an die Verwendung eines \emph{Reverse-Engineering}-Schutzes gedacht. Sowohl die \texttt{Horde3DStateWatcher}-Klasse als auch die Anforderung vor dem Schutz vor unerw�nschtem \emph{Reverse-Engineering} wurden erst in einer sp�teren Iteration ins \DevEnv\ aufgenommen und f�gten sich nahtlos in das Systemdesign ein.

Die Entwicklung des GUI-Frameworks war zeitaufw�ndig und h�tte vermieden werden k�n\-nen, wenn das \DevEnv\ als Plugin f�r Visual Studio oder SharpDevelop entwickelt worden w�re. Aufgrund verschiedener Unzul�nglichkeiten der Plugin-Infrastruktur der IDEs hat sich die Eigenentwicklung schlie�lich doch als die bessere L�sung herausgestellt, da sich w�hrend der Implementierung des Systems die St�rken des Frameworks zeigten und ein z�gige und unkomplizierte Umsetzung des Designs erm�glichten. So ist der GUI- und Anwendungscode stets klar voneinander abgegrenzt und es ist einfach, neue Features durch Implementieren weiterer \texttt{Presenter}- und \texttt{View}-Klassen hinzuzuf�gen. Die Wiederverwendbarkeit und Erweiterbarkeit des Frameworks konnte bereits im Rahmen der Bachelorarbeit �berpr�ft werden. So wurde zu einem sp�teren Zeitpunkt eine weitere \texttt{DockView}-Subklasse, \texttt{WpfDockView}, hinzugef�gt, mit der Windows Presentation Foundation \texttt{UserControl}s in Windows Forms \texttt{DockView}s dargestellt werden k�nnen. Der Einsatz des Frameworks bei der Entwicklung des Code Generators, siehe Abschnitt~\ref{CodeGen}, best�tigte die Wiederverwendbarkeit der Bibliothek.