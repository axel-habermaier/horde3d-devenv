\chapter{Inhalt der beiliegenden CD-ROM}
Auf der beiliegenden CD-ROM dieser Bachelorarbeit befinden sich einige der in der Arbeit referenzierten Dokumente und Anwendungen sowie der Source Code des \DevEnvs.

\begin{itemize}
	\item \texttt{Bachelorarbeit.pdf} ist diese Bachelorarbeit im .pdf-Format.
	
	\item \texttt{Designmodell.pdf} enth�lt das Designmodell des \DevEnvs, das aufgrund seiner Gr��e in der ausgedruckten Arbeit nicht abgebildet werden konnte.
	
	\item In der Datei \texttt{Frageb�gen.pdf} sind alle vier ausgef�llten Frageb�gen (anonym) aufgelistet.
	
	\item \texttt{Konzeptmodell.pdf} enth�lt das Konzeptmodell des \DevEnvs.
	
	\item Im Verzeichnis \texttt{Horde3D DevEnv} befindet sich der \emph{Installer} f�r das \DevEnv. Die Installationsroutine wird durch das Ausf�hren der Datei \texttt{Horde3D DevEnv.msi} gestartet. Alle ben�tigten \emph{Dependencies} (.NET 3.5 SP1 und Visual Studio \emph{Redistributable} x86) sollten automatisch installiert werden. Nach der Installation kann das \DevEnv\ sofort gestartet werden. Es liegt das \emph{Knight Sample} aus dem \Horde\ SDK bei, das bereits vorkonfiguriert ist und gestartet werden kann.
	
	\item Im Verzeichnis \texttt{lwar} befindet sich ein \emph{Debug Build} des Raumschiff-Spiels aus Kapitel~\ref{space}. In dieser Version kann das Raumschiff nicht bewegt werden; durch Dr�cken der Tasten $j,h,g,z$ k�nnen Treffer von rechts, unten, links und oben simuliert werden. Der Shader \texttt{shields.shader.xml}, der im Rahmen von Kapitel~\ref{space} entwickelt wurde, ist im Verzeichnis \texttt{lwar/Content/effects/shields} zu finden. 
	
	\item Ein \emph{Debug Build} des \SheepMeUp-Ports auf \Horde\ 1.0.0 Beta 3 befindet sich im Verzeichnis \texttt{SheepMeUp}. Wie in Kapitel~\ref{smuport} beschrieben, ist diese Version des Spiels zwar lauff�hig, aber nicht spielbar. Gestartet werden kann das Spiel �ber die Datei \texttt{SheepMeUp.bat}.
	
	\item Das Verzeichnis \texttt{Source Code} enth�lt den kompletten Source Code des \DevEnvs. Durch �ffnen der \emph{Solution} \texttt{Bachelorarbeit.sln} kann das \DevEnv\ mit Visual Studio 2008 SP1 bearbeitet werden. Zum Kompilieren, Debuggen und Ausf�hren der Anwendung sind die Hinweise aus Kapitel~\ref{aufbau} zu beachten.
	
	\item Im Verzeichnis \texttt{Websites} befinden sich Kopien der im Rahmen der Bachelorarbeit besuchten Websites. Das jeweilige Aufrufsdatum stimmt mit dem im Literaturverzeichnis angegebenen Datum �berein.
	
\end{itemize}