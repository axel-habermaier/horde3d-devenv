\section{Konklusion}
Es stellte sich als die richtige Entscheidung heraus, die Funktionsweise und den Aufbau von \Horde\ vor der eigentlichen Anforderungsanalyse genau zu untersuchen. Dadurch konnten wichtige Einblicke in den Problembereich gewonnen werden, da das Konzeptmodell fast komplett vorgegeben wurde. Zusammen mit den Erfahrungen aus der Entwicklung von SheepMeUp konnten viele Systemanforderungen gefunden werden, die sp�ter als hilfreich und wichtig best�tigt wurden.

%Das Konzeptmodell war eine gro�e Hilfe in der Design-Phase und wurde in sp�teren Iterationen fast nicht ver�ndert. Die Anforderungen an das System wurden im Design und in der Implementierung vollst�ndig umgesetzt und erwiesen sich als �u�ert hilfreich bei der Entwicklung eines Hitzeschimmer-Effekts f�r \SheepMeUp.

Die Betrachtung �hnlicher und verwandter Softwaretools best�tigte au�erdem, dass es f�r \Horde\ kein zum \DevEnv\ vergleichbares Tool gibt. Des Weiteren konnten einige Ideen der Tools in die Anforderungen, das Design oder die Implementierung des Systems integriert werden.