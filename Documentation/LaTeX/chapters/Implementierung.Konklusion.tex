\section{Konklusion}
F�r die Implementierung des \DevEnvs\ wurde eine Mischung aus nativem \C++, \C++/CLI und \Csharp\ verwendet. In vielen Bereichen konnten bereits vorhandene Bibliotheken gewinnbringend verwendet werden; insbesondere wurde die Umsetzung der GUI durch die Verwendung der Weifen Luo Winforms Docking Bibliothek und einiger anderer \emph{Controls} deutlich erleichtert. 

Insgesamt wurden weite Teile des Designs erfolgreich umgesetzt. Es wurde jedoch nur der Code entwickelt, der zum Erf�llen der Anforderungen erforderlich war. Gerade bei den \Horde-Klassen sind das Konzept- und auch das Designmodell jedoch detaillierter, als die Anforderungen eigentlich verlangten. An diesen Stellen wurden nur die erforderlichen Teile umgesetzt; die fehlenden Codeteile k�nnen jederzeit erg�nzt werden.

Beim Auslesen der Ressourcen-Daten fiel ein Problem mit der \Horde-API auf: Es gab keine einfache M�glichkeit, alle \Horde\ derzeit bekannten Ressourcen zu ermitteln. Jedoch wurde auf Nachfrage bei den Entwicklern eine entsprechende Funktion in \Horde\ 1.0.0 Beta 3 eingef�hrt. Die API wurde in dieser Version zus�tzlich um eine Funktion erg�nzt, die den Abschluss eines Frames markiert. Der Profiling-Mechanismus ist auf das Vorhandensein dieser Funktion angewiesen. Somit sind diese beiden Funktionen der Grund, warum das \DevEnv\ nur mit Beta 3 kompatibel ist.

Die Implementierung des GUI-Frameworks ben�tigte gegen�ber dem Designmodell einige zus�tzliche Klassen. Die neuen Klassen waren jedoch nur notwendig, um die \texttt{Shell}-, \texttt{Presenter}- und \texttt{IView}-Klassen durch Verwendung von \emph{Generics} typsicher zu machen. Das erm�glichte ein schnelleres Umsetzen der notwendigen \emph{Presenter} und \emph{Views} des \DevEnvs.

Die Erstellung des Code Generators nahm einige Zeit in Anspruch, da zun�chst die Analyse- und Design-Phasen durchlaufen werden mussten. Es zeigte sich jedoch, dass rund 4800 Zeilen Code automatisch generiert werden konnten und nicht von Hand eingetippt werden mussten. Da der Code Generator auch gut mit Updates der \Horde-API zurecht kommt, hat sich auch die detaillierte Analyse der Anforderungen und der m�glichen Probleme bei der Code Generierung gelohnt. Durch die Verwendung des GUI-Frameworks bei der Implementierung der Benutzeroberfl�che des Code Generators hielt sich der Implementierungsaufwand in Grenzen. Zudem konnten fr�h erste Erfahrungen im Umgang mit dem GUI-Framework gesammelt werden und einige Unsch�nheiten bei der Typsicherheit, der Sichtbarkeit von eigentlich internen \emph{Properties} sowie dem thread-sicheren Zugriff auf die \emph{View} eines \emph{Presenters} behoben werden.